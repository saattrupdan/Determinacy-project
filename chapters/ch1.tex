\chapter{Basic game theory}
\thispagestyle{fancy}
\label{ch1}

\section{Infinite games}
Let $X$ be any set and $A\subset{^\omega X}$. To such a pair we can associate a game $G_X(A)$ between two players I and II, where they alternate playing elements $x_i\in X$ for $i<\omega$:
\game{x_0}{x_1}{x_2}{x_3}{x_4}{x_5}{\cdots}{\cdots}

\qquad This then results in an element $(x_i)\in{^\omega X}$ called a \textbf{play} and initial segments of a play are called \textbf{partial plays}. For $x$ a play, we write $x_I\in{^\omega X}$ for the sequence of player I's moves in $x$, and likewise $x_{II}\in{^\omega X}$ for the sequence of player II's moves in $x$. We say that player I \textbf{wins} if $(x_i)\in A$ and player II wins otherwise - $A$ is called the \textbf{payoff set}.

\qquad A \textbf{strategy} for player I in $G_X(A)$ is a function $\sigma:\bigcup_{n<\omega}{^{2n}X}\to X$ and a strategy for player II is a function $\tau:\bigcup_{n<\omega}{^{2n+1}X}\to X$. We think of a strategy (for e.g. player I) as a way to provide moves for player I given how player II has played so far. Thus If $\sigma$ is a strategy for player I, the game is played as follows
\game{\sigma(\bra{})}{y_0}{\sigma(\bra{\sigma(\bra{}),y_0})}{y_1}{\cdots}{\cdots}{}{}

\qquad We denote such a play by $\sigma*y\in {^\omega X}$. A \textbf{winning strategy} for player I is a strategy $\sigma$ for player I such that $\{\sigma*y\mid y\in{^\omega X}\}\subset A$, i.e. that no matter what player II does, player I will win if he follows the strategy. A winning strategy for player II is defined analogously.

\qquad A game is \textbf{determined} if one of the players has a winning strategy. We call games of the type $G_X(A)$ \textbf{perfect information games}. We say that two games $G,G'$ are \textbf{equivalent} if player I has a winning strategy in $G$ iff he has one in $G'$, and player II has a winning strategy in $G$ iff he has one in $G'$.

\qquad Putting the discrete topology on $X$ and the product topology on ${^\omega X}$, we call a game $G_X(A)$ \textbf{open, closed, Borel, analytic} etc. if the payoff set $A\subset {^\omega X}$ is open, closed, Borel, analytic etc.

\qquad We also have a notion of \textbf{games with rules}. Letting $T\subset{^\omega X}$ be a pruned tree, we restrict the moves allowed by the players to those $x\in X$ such that the partial play is an element of $T$. For instance, if a partial play looks like $\bra{x_0,\hdots,x_n}$, then $x\in X$ is a \textbf{legal move} if $\bra{x_0,\hdots,x_n,x}\in T$. Now $G_X(T,A)$ is then the game $G_X(A)$ with legal moves $T$. If we don't care about the payoff set, then we can leave it out and just write $G_X(T)$.

\qquad Strategies $\sigma,\tau$ are defined as before with the extra requirement that $\sigma(s),\tau(t)$ are legal for all $s,t$. We say $y\in{^\omega X}$ is \textbf{$\sigma$-legal} if $\sigma*y\in[T]$ and $\sigma$ is \textbf{winning} if $\{\sigma*y\mid\godel{y\text{ is $\sigma$-legal}}\}\subset A$, and analogously for $\tau$. 

\qquad It turns out the we're not obtaining a larger class of games by introducing games with rules, because the game $G_X(T,A)$ can be seen to be equivalent to the game $G_X(B)$, where $B$ is defined as the set
\eq{
([T]\cap A)\cup\{x\in{^\omega X}\mid\exists n(x\restr n\notin T)\land\godel{\min\{n<\omega\mid x\restr n\notin T\}\text{ is even}}\}.
}

\section{Regularity properties and games}

One of the reasons why one might be interested in whether or not games are determined can be seen by how determinacy of certain games entails regularity properties of sets of reals. In this section we'll introduce three games, which are closely related to whether or not a set has the Baire property, the perfect set property or is Lebesgue measurable.

\qquad We start off with examining the Baire property.

\defi{
Let $A\subset\n$. Define the \textbf{Banach-Mazur game} $G^{**}_\omega(A)$ as
\game{s_0}{s_1}{s_2}{s_3}{s_4}{s_5}{\cdots}{\cdots}

where $s_i\in\tree-\{\bra{}\}$. Here player I wins iff $s_0 \cc s_1\cc s_2\cc \hdots\in A$.
}

\pagebreak
\lemm[Banach, Mazur]{
\label{** lemma}
Let $A\subset\n$. Then in the game $G^{**}_\omega(A)$:
\begin{enumerate}
\item Player II has a winning strategy iff $A$ is meager.
\item Player I has a winning strategy iff there is some $s\in\tree$ such that $N_s-A$ is meager;
\end{enumerate}
}
\proof{
$(i):$ Assume player II has a winning strategy $\tau$. For each partial play consistent with $\tau$ of the form $p=\bra{s_0,s_1,\hdots,s_{2n-1}}$, set $p_*:=s_0\cc s_1\cc\cdots\cc s_{2n-1}$ and
\eq{
D_p:=\{x\in\n\mid p_*\subset x\to(\exists t\in\tree-\{\bra{}\})(p_*\cc t\cc\tau(p\cc\bra{t})\subset x)\}.
}

\clai{
$D_p$ is open and dense.
}

\cproof{
We have that
\eq{
D_p=\lnot N_{p_*}\cup\bigcup_{t\in\tree-\{\bra{}\}}N_{p_*\cc t\cc\tau(p\cc\bra{t})},
}

so as the basic open sets are clopen, $D_p$ is open. As for denseness, let $u\in\tree$. If $p_*$ and $u$ are incomparable then $N_u\subset D_p$ vacously. If $u\subset p_*$ then pick $k$ such that $u\cc\bra{k}\nsubset p_*$ and thus $N_{u\cc\bra{k}}\subset D_p$ vacously. If $p_*\subset u$ then there is some $t\in\tree-\{\bra{}\}$ such that $p_*\cc t=u$. Now pick any $x\in\n$ such that $p_*\cc t\cc\tau(p\cc\bra{t})\subset x$ whence $x\in N_u\cap D_p$, making $D_p$ dense.
}

Hence $\bigcup_p\lnot D_p$ is meager. We'll show $A\subset\bigcup_p\lnot D_p$. If $x\in\bigcap_p D_p$ then we can recursively define a play $(s_i)$ compatible with $\tau$ such that $x=s_0\cc s_1\cc\cdots$, making $x\notin A$ as $\tau$ is winning. Thus $A\subset\bigcup_p\lnot D_p$, concluding $A$ is meager.

\qquad Now assume conversely that $A$ is meager, so $A\subset\bigcup_{n<\omega}C_n$ with $C_n\subset\n$ being closed and nowhere dense. Then we can define a strategy $\tau$ for player II as follows. Set $\tau(\bra{s_0})$ to be some $t$ such that $N_{s_0\cc t}\cap C_0=\emptyset$, which is possible as $C_0$ is nowhere dense. Then recursively set $\tau(\bra{s_0,\hdots,s_{2n}})$ to be some $t$ such that $N_{s_0\cc\cdots\cc s_{2n}\cc t}\cap C_n=\emptyset$. It's easily seen that such a strategy is winning for player II.\\

$(ii):$ Assume first that player I has a winning strategy $\sigma$ and set $s:=\sigma(\bra{})$. Then it's easily seen that player II has a winning strategy in the game $G_\omega^{**}(N_s-A)$, derived from $\sigma$. Hence by $(i)$, $N_s-A$ is meager.

\qquad Assume now that there is some $s\in\tree$ such that $N_s-A$ is meager. If $s=\bra{}$ then $A$ is comeager and hence $A$ is dense by \ref{baire cat theo}, so player I clearly has a winning strategy. Assume thus $s\neq\bra{}$. Then the strategy $\sigma$ given by $\sigma(\bra{})=s$ and then avoiding $N_s-A$ as we did in $(i)$, is winning.
}

\prop{
\label{bairegame prop}
Let $A\subset\n$ and define
\eq{
O_A:=\bigcup\{N_s\mid s\in\tree\land N_s-A\text{ is meager}\}.
}

If $G^{**}_\omega(A-O_A)$ is determined then $A$ has the Baire property.
}
\proof{
Assume player I has a winning strategy. Then by Lemma \ref{** lemma} there is some $s\in\tree$ such that $N_s-(A-O_A)$ is meager. Hence $N_s-A$ is also meager, so that $N_s\subset O_A$. But then $N_s-(A-O_A)=N_s$, which cannot be meager, $\contr$.

\qquad Hence player II must have a winning strategy, making $A-O_A$ meager by Lemma \ref{** lemma}. As we also have that $O_A-A$ is meager by definition of $O_A$ and that a union of meager sets is meager, $A=_*O_A$.
}

As for the perfect set property, we turn to a game constructed by Davis.

\defi{
Let $A\subset\c$. Then define the \textbf{Davis game} $G^*_2(A)$ as
\game{s_0}{x_0}{s_1}{x_1}{s_2}{x_2}{\cdots}{\cdots}

where $s_i\in{^{<\omega}2}$ and $x_i\in 2$. Then player I wins iff $s_0\cc \bra{x_0}\cc s_1\cc \bra{x_1}\cc\cdots \in A$.
}

\lemm[Davis]{
\label{* lemma}
Let $A\subset\c$. Then in the game $G^*_2(A)$:
\begin{enumerate}
\item Player I has a winning strategy iff $A$ has a perfect subset;;
\item Player II has a winning strategy iff $A$ is countable.
\end{enumerate}
}
\proof{
$(i):$ Assume $\sigma$ is a winning strategy for player I. We claim that
\eq{
P:=\{\sigma*y\mid y\in\c\}
}

is a perfect subset of $A$. It's a subset of $A$ since $\sigma$ is winning. Let $x\in\lnot P$. Then there is some $n<\omega$ such that $x\restr n\neq(\sigma*y)\restr n$ for every $y\in\c$. But then $N_{x\restr n}\cap P$ is an open neighborhood around $x$, so $\lnot P$ is open, making $P$ closed.

\qquad Lastly let $\sigma*y\in P$ and $N_s\subset\c$ an open basis neighborhood around $\sigma*y$. If $\len(s)$ is even then let player II play any other move than in $\sigma*y$ and play arbitrarily following $\sigma$, resulting in $\sigma*y'\in P\cap N_s$. If $\len(s)$ is odd then $\len(s\cc\bra{\sigma(s)})$ is even, so repeat the previous argument. Hence $P$ is perfect in $A$.

\qquad Assume conversely that $P\subset A$ is a perfect subset. Defining
\eq{
T:=\{x\restr n\mid x\in P\land n<\omega\},
}

we can form a strategy $\sigma$ for player I as follows. Set $\sigma(\bra{})$ to be $s\in T$ such that both $s\cc\bra{0}\in T$ and $s\cc\bra{1}\in T$, which can be done since $P$ has no isolated points. In general in response to a partial play $p$, set $\sigma(p)$ to be $s\in T$ such that both $s\cc\bra{0}\in T$ and $s\cc\bra{1}\in T$. This strategy is easily seen to be winning for player I.\\

$(ii):$ Assume $\tau$ is a winning strategy for player II. Arguing like in Lemma \ref{** lemma}, for a partial play of the form $p:=\bra{s_0,x_0,\hdots,x_{2n-1}}$ set $p_*:=s_0\cc\bra{x_0}\cc\cdots\cc\bra{x_{2n-1}}$ and define
\eq{
D_p:=\{x\in\c\mid p_*\subset x\to(\exists t\in{^{<\omega} 2})(p_*\cc t\cc\tau(p\cc\bra{t})\subset x)\}.
}

As previously, we have that $A\subset\bigcup_p\lnot D_p$. We'll show that $\lnot D_p$ contains exactly one element, making $A$ countable. We have that $x\in\lnot D_p$ iff $p_*\subset x$ and $(\forall t\in{^{<\omega}2})(p_*\cc t\cc\tau(p\cc\bra{t})\nsubset x)$, so say $\len(p_*)=m$ and we have some $x\in\c$ such that $x\restr m=p_*$. Now set $i:=\tau(p\cc\bra{\emptyset})\in 2$. Then we necessarily have that $x(m)=1-i$ since otherwise $p_*\cc\emptyset\cc\tau(p\cc\bra{\emptyset})\subset x$. Likewise, recursively we must have that
\eq{
x(e):=1-\tau(p\cc\bra{x(m),\hdots,x(e-1)})
}

for $e>m$. Hence $\lnot D_p$ has exactly one element $x$, making $A$ countable.

\qquad Now on the other hand assume that $A$ is countable; write $A:=\{a_i\mid i<\omega\}$. Player II can now just diagonalize $A$, playing his $i$'th move to make sure that that the partial play differs from $a_i$. This strategy is clearly winning for player II.
}

We have a canonical way of going back and forth between $\n$ and $\c$:

\defi{
For $n,k<\omega$ define the functions $b_n^k:n+1\to 2$ as
\eq{
b_n^{2k}(i):=\left\{\begin{array}{ll}1 &,i<n\\ 0 &,i=n \end{array}\right.
\qquad
b_n^{2k+1}(i):=\left\{\begin{array}{ll}0 &,i<n\\ 1&,i=n \end{array}\right.
}

Then define the function $\Psi:\n\to\c$ as $\Psi(x):=b_{x(0)}^0\cc b_{x(1)}^1 \cc b_{x(2)}^2 \cc\hdots$
}

For instance, if $x=\bra{1,2,3,\hdots}$ then
\eq{
\Psi(x)=b_1^0\cc b_2^1\cc b_3^2\cc\hdots=\bra{1,0,0,0,1,1,1,1,0,\hdots},
}

which we also suggestively could write as $1^100^211^30\cdots$.

\prop{
\label{homeo prop}
$\Psi:\n\to\c$ is a homeomorphism onto $\c_0$, where $\c_0\subset\c$ is the elements that are not eventually constant.
}
\proof{
Just as in the example above, notice that every $x\in\c_0$ can be written as
\eq{
x=1^{\alpha_0}00^{\alpha_1}11^{\alpha_2}0\cdots=b^0_{\alpha_0}\cc b^1_{\alpha_1}\cc b^2_{\alpha_2}\cc\cdots
}

for $\alpha_i<\omega$. Define $\Phi:\c_0\to\n$ given by $\Phi(b^0_{\alpha_0}\cc b^1_{\alpha_1}\cdots):=\bra{\alpha_0,\alpha_1,\hdots}$. Then $\Phi$ is clearly an inverse to $\Psi$. It remains to show that both $\Psi$ and $\Phi$ are continuous. But we have that
\eq{
&\Psi^{-1}(N_{b^0_{\alpha_0}\cc b^1_{\alpha_1}\cc\cdots\cc b^n_{\alpha_n}}\cap\c_0)=N_{\bra{\alpha_0,\alpha_1,\hdots,\alpha_n}}\\
&\Phi^{-1}(N_{\bra{\alpha_0,\alpha_1,\hdots,\alpha_n}})=N_{b^0_{\alpha_0}\cc b^1_{\alpha_1}\cc\cdots\cc b^n_{\alpha_n}}\cap\c_0,
}

so both are continuous, and $\Psi$ is thus a homeomorphism.
}

\prop{
\label{perfectsetgame prop}
For $A\subset\n$, $G^*_2(\Psi"A)$ is determined iff $A$ has the perfect set property.
}
\proof{
By Lemma \ref{* lemma}, we only have to show that $A\subset\n$ is countable iff $\Psi"A\subset\c$ is countable, and $A\subset\n$ is perfect iff $\Psi"A\subset\c$ is perfect. But this is clear by the previous Proposition \ref{homeo prop}.
}

Lastly we deal with Lebesgue measurability. The game turns out to be more complicated, even though the idea is very simple.

\defi{
For $\varepsilon>0$ and $A\subset\n$, define the \textbf{covering game} as the game $G(A,\varepsilon)$ given by
\game{x_0}{y_0}{x_1}{y_1}{x_2}{y_2}{\cdots}{\cdots}

where $x_i\in 2$ and $y_i\in\omega$. Letting $\{s_i\mid i<\omega\}=\tree$ be an enumeration of $\tree$, set $U_i:=N_{s_{s_i(0)}}\cup\cdots\cup N_{s_{s_i(\len(s_i)-1)}}$. We require of player I that $(x_i)$ is not eventually constant and of player II that $\lambda(U_{y_i})<\varepsilon/2^{2(i+1)}$ -- if either player cannot satisfy this, he loses. If this doesn't happen then player I wins iff $\Psi^{-1}(x)\in A-\bigcup_{n<\omega}U_{y_n}$. I.e. player II tries to cover $A$ with $\bigcup_{n<\omega}U_{y_n}$.
}

\lemm{
\label{cov lemma}
Let $A\subset\n$ and $\varepsilon>0$. Then in the covering game $G(A,\varepsilon)$:
\begin{enumerate}
\item If player I has a winning strategy then there's a Lebesgue measurable $B\subset A$ with positive Lebesgue measure;
\item If player II has a winning strategy then there's an open set $O\supset A$ such that $\lambda(O)<\varepsilon$.
\end{enumerate}
}
\proof{
$(i):$ Let $\sigma$ be a winning strategy for player I. Define 
\eq{
B:=\{\Psi^{-1}((\sigma*y)_I)\mid y\in\n\}.
}
Since $\sigma$ is winning, $\Psi^{-1}((\sigma*y)_I)\in A$ for every $y\in\n$, so $B\subset A$. Define the function $f:\n\to\n$ as $f(y):=\Psi^{-1}((\sigma*y)_I)$. We claim that $f$ is continuous. It's enough to check that $f_n:\n\to 2$ given by $f_n(y):=(\sigma*y)_I(n)=\sigma((\sigma*y)\restr 2n)$ is continuous for every $n<\omega$. And indeed, we have that
\eq{
f_n^{-1}(\{k\})&=\{y\in\n\mid(\sigma*y)_I(n)=k\}\\
&=\bigcup\{N_s\mid s\in\tree\land\len(s)=2n\land\godel{s\text{ follows }\sigma}\land\sigma(s)=k\}
}

is open, so $f_n$ is continuous, making $f$ continuous as well. This means that $B$ is a continuous image of the Baire space, making it analytic and hence Lebesgue measurable by Theorem \ref{analytic meas theo}.

\qquad It remains to show that $B$ has positive Lebesgue measure. Assume $B$ is $\lambda$-null. By Lemma \ref{0.9 lemma}, there exists an open $O\supset B$ such that $\lambda(O)<\varepsilon/4=\varepsilon/2^{2(0+1)}$. Since $O=\coprod_{i<\omega} N_{s_i}$ by Proposition \ref{0.8 prop}, cut the $N_{s_i}$'s into smaller basic opens to wind up with a countable collection of basic opens $(N_{s_j})_{j<\omega}$ such that $O=\coprod_{j<\omega}N_{s_j}$ and $\lambda(N_{s_j})<\varepsilon/2^{2(j+1)}$ for every $j<\omega$. Then the $y\in\n$ coding $(N_{s_j})_{j<\omega}$ is a winning strategy for player II, $\contr$. Hence $\lambda(B)>0$.\\

$(ii):$ Let $\tau$ be a winning strategy for player II. For any $s\in{^{<\omega}2}-\{\bra{}\}$, say with $\len(s)=n+1$, set $d(s)\in X$ to be player II's response to the partial play consistent with $\tau$ where player I's moves are $s(0),\hdots,s(n)$. Since $\tau$ is winning then given any $x\in A$ there exists some $n<\omega$ such that $x\in U_{d(\Psi(x)\restr n+1)}$. Thus, defining
\eq{
O:=\bigcup\{U_{d(s)}\mid s\in\tree-\{\bra{}\}\},
}

we see that $A\subset O$. The rules of game require that $\lambda(U_{d(s)})<\varepsilon/2^{2(n+1)}$ for $\len(s)=n+1$, so since there are $2^{n+1}$ such $s$'s, we have that
\eq{
\lambda(O)<\sum_{n<\omega}\frac{\varepsilon}{2^{2(n+1)}}\cdot 2^{n+1}=\sum_{n<\omega}\frac{\varepsilon}{2^{n+1}}=\varepsilon.
}
}

\defi{
Let $A,B\subset\n$. Then $B$ is a \textbf{minimal cover} for $A$ if $A\subset B$, $B$ is Lebesgue measurable and if $Z\subset B-A$ is a Lebesgue measurable set then $\lambda(Z)=0$.
}

Notice that any $A\subset\n$ has a minimal $G_\delta$ cover by Lemma \ref{0.9 lemma}.

\prop{
\label{lebesguegame prop}
Let $A\subset\n$ and let $B\subset\n$ be a minimal $G_\delta$ cover for $A$. If $G(B-A,\varepsilon)$ is determined for every $\varepsilon>0$ then $A$ is Lebesgue measurable.
}
\proof{
Assume $G(B-A,\varepsilon)$ is determined for every $\varepsilon>0$. By Lemma \ref{cov lemma}, it's impossible for player I to have a winning strategy as $B$ is a minimal cover for $A$. Hence by the same Lemma there exist open sets $O_\varepsilon\supset B-A$ with $\lambda(O_\varepsilon)<\varepsilon$ for every $\varepsilon>0$. Defining
\eq{
C:=\bigcap_{n<\omega}O_{1/n},
}

we see that $B-A\subset C$ and $\lambda(C)=0$. Hence $A\triangle B$ is a $\lambda$-null set, so since $B$ is $G_\delta$ and in particular Borel, $A$ is Lebesgue measurable.
}

\section{Axiom of determinacy}

Mycielski and Steinhaus proposed the following axiom in \cite{AD}.

\defi{
The \textbf{Axiom of Determinacy}, $\ad$, is the statement that $G_\omega(A)$ is determined for every $A\subset\n$, i.e. that every perfect information game of reals is determined.
}

\theo[$\zf$]{
\label{ad is nice theo}
$\ad$ implies that every subset of the reals is Lebesgue measurable and has both the Baire property and the perfect set property.
}
\proof{
This follows by Propositions \ref{bairegame prop}, \ref{perfectsetgame prop} and \ref{lebesguegame prop} if we can show that the three corresponding games are equivalent to a game of the form $G_\omega(B)$. We start off with the Banach-Mazur game $G^{**}_\omega(A)$, where each player playes sequences $s_i\in\tree-\{\bra{}\}$ and player I won iff $s_0\cc s_1\cc\cdots\in A$. By fixing an enumeration $\{s_i\mid i<\omega\}$ of $\tree-\{\bra{}\}$, the players can just play $x_i<\omega$ instead and setting $B:=\{x\in\n\mid s_{x(0)}\cc s_{x(1)}\cc\cdots\in A\}$, $G^{**}_\omega(A)$ is equivalent to $G_\omega(B)$.

\qquad Now for the Davis game $G^*_2(A)$, where player I played $s_i\in{^{<\omega}2}$, player II played $x_i\in 2$ and player I won iff $s_0\cc\bra{x_0}\cc s_1\cc\bra{x_1}\cc\cdots\in A$. By fixing an enumeration $\{s_i\mid i<\omega\}$ of ${^{<\omega}2}$, player I and II can play $x_i<\omega$ and $y_i<\omega$, respectively, and setting
\eq{
B:=\{x\in\n\mid s_{x_0}\cc\bra{\Psi(x_1)}\cc s_{x_2}\cc\bra{\Psi(x_3)}\cc\cdots\in A\},
}

we see that $G^*_2(A)$ is equivalent to $G_\omega(B)$. Hence we also have that $G^*_2(\Psi"A)$ is equivalent to $G_\omega(\Psi"B)$.

\qquad Lastly, for the covering game $G(A,\varepsilon)$ we can just make player I play $x_i\in\omega$ instead and say that player II wins iff $x\in A-\bigcup_{n<\omega}N_{y_n}$.
}

We see that $\ad$ implies a lot of nice properties about the real numbers, but these properties turns out to be a bit \textit{too} nice: $\ad$ contradicts the axiom of choice.

\coro{
$\ac$ implies that $\ad$ is false.
}
\proof{
As $\ac$ allows us to construct a set not having the Baire property by Proposition \ref{bernstein prop}, we have that $\zf\proves\ac\to\lnot\ad$.
}

Hence we can see $\ad$ as removing the ``non-regular'' sets of reals. $\ad$ and $\ac$ are not completely inconsistent however; we're allowed to accept \textit{countable} choice over the reals $\ac_\omega(\n)$, stating that every countable set of non-empty sets of reals has a choice function.

\prop[$\zf$]{
$\ad$ implies $\ac_\omega(\n)$.
}
\proof{
Let $\{X_n\mid n<\omega\}\subset\mathcal{P}(\n)-\{\emptyset\}$. Define
\eq{
A:=\{x\in\n\mid x_{II}\notin X_{x(0)}\}.
}

Clearly player I cannot have a winning strategy in $G_\omega(A)$, so $\ad$ implies that we have a winning strategy $\tau$ for player II. Define now $f:\omega\to\n$ given by $f(n):=(\bra{n,0,0,\hdots}*\tau)_{II}$. It's easy to check that $f$ is a choice function.
}


